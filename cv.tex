% Current version on https://www.overleaf.com/project/67c5b769ac7a184967b24914

\documentclass[11pt,letterpaper,roman]{moderncv}
\moderncvstyle{classic}
\moderncvcolor{blue}

% Adjust page margins for better space utilization
\usepackage[scale=0.75]{geometry}
\setlength{\footskip}{136.00005pt}

% Datos personales
\name{Omar}{Olivares Urrutia}
\title{Ingeniero de Inteligencia Artificial}
\address{Providencia, Chile}
\phone[mobile]{+56~9~8381~0103}
\email{omar@olivares.cl}
\social[linkedin]{ofou}
\homepage{olivares.cl}
\photo[64pt][0.4pt]{photo.jpeg}

\begin{document}

\makecvtitle

% Resumen Profesional
\section{Resumen Profesional}
\cvitem{}{Ingeniero de Software en \href{https://emergentmind.com}{Emergent Mind}, especializado en Inteligencia Artificial, Machine Learning y aplicaciones. He implementado sistemas RAG y fine-tuning para mejorar la calidad de las respuestas de modelos de lenguaje. Como divulgador científico y creador de contenidos independiente (en inglés), he producido videos virales que han superado las 200.000 vistas sobre contenidos técnicos acerca de interfaces cerebro-computador.}

% Educación (movida antes de experiencia laboral)
\section{Educación}
\cventry{2018 -- 2024}{Ingeniería Civil Informática}{\href{https://ucm.cl}{Universidad Católica del Maule}}{}{}{\textit{Especialización en procesamiento de lenguaje natural e Inteligencia Artificial.} \\ Minor en Composición Musical. Graduado con honores.}
\cventry{2015 -- 2018}{Ingeniería Civil en Computación}{\href{https://utalca.cl}{Universidad de Talca}}{}{}{\textit{Enfoque en arquitectura de computadores, algoritmos y desarrollo de software.} \\ No finalizada, por cambio de Universidad.}

% Publicaciones (relacionado con la educación y logros académicos)
\section{Publicaciones}
\cvitem{Título}{\emph{Optimización de Modelos de Lenguaje para la Prueba de Acceso a la Educación Superior (PAES) de Matemáticas en Chile: Una prueba de concepto}}
\cvitem{Supervisores}{
  \href{https://www.linkedin.com/in/xaviera-lopez-cortes}{Dra. Xaviera A. Lopez Cortes}, \href{https://www.linkedin.com/in/maria-aravena-díaz}{Dra. Maria D. Aravena Diaz} y \href{https://investigadores.anid.cl/es/public_search/researcher?id=9261}{Dr. Sergio I. Hernández Alvarez}
}

% Experiencia Laboral (tras educación y publicaciones académicas)
\section{Experiencia Laboral}
\cventry{Febrero 2024 -- Presente}{Ingeniero de Inteligencia Artificial}{\href{https://www.emergentmind.com}{Emergent Mind}}{Carolina del Norte, Estados Unidos}{}{%
\begin{itemize}%
\item Diseño pipelines con RAG para mejorar la calidad y relevancia de los resultados de búsqueda en Azure.
\item Colaboro con el fundador en la estrategia y desarrollo del producto para la plataforma de descubrimiento de investigaciones en arXiv.
\item Implemento y optimizo modelos de aprendizaje profundo y modelos de lenguaje.
\item Realizo experimentos para optimizar el rendimiento y la precisión de los modelos.
\item Desarrollo evaluación y control de calidad para modelos de lenguaje que utilizan RAG.
\item Conceptualizo y prototipo características innovadoras de IA para impulsar el crecimiento de la plataforma.
\end{itemize}}

\cventry{Septiembre 2021 -- Enero 2025}{Desarrollador de Contenido Técnico}{\href{https://youtube.com/@NeuraPod}{NeuraPod}}{San Diego, Estados Unidos}{}{%
\begin{itemize}%
\item Produje contenido sobre neurotecnología e IA, contribuyendo al crecimiento del canal a más de 80.000 suscriptores en YouTube.
\item Gestioné la producción completa: ideación, guiones, grabación de locuciones, edición y distribución de videos.
\item Traduje conceptos científicos complejos en contenido atractivo y accesible para audiencias amplias.
\item Creé múltiples videos virales con más de 200.000 vistas cada uno.
\item Expandí la presencia de Neura Pod en plataformas de redes sociales.
\end{itemize}}

\cventry{Julio 2021 -- Septiembre 2021}{Pasante}{\href{https://youtube.com/@NeuraPod}{Neura Pod}}{San Diego, Estados Unidos}{}{Inicié mi carrera en creación de contenido técnico, aprendiendo las bases de la producción audiovisual y la comunicación científica.}

\cventry{2021}{Ingeniero de Software}{\href{https://www.rayo.cl}{Rayo}}{Remoto}{}{Optimicé el uso de MongoDB y SQL, reduciendo costos operativos en un 15\% mediante técnicas avanzadas de indexación y caching.}

\cventry{2020}{Ingeniero de Software}{\href{https://www.suncast.cl}{Suncast}}{Las Condes, Chile}{}{Ayudé a reconstruir la plataforma en una arquitectura de microservicios, mejorando la escalabilidad y reduciendo el tiempo de respuesta en AWS.}

\cventry{2016 -- 2018}{Asistente de Soporte en TI}{\href{https://www.utalca.cl}{Universidad de Talca}}{Curicó, Chile}{}{Resolví problemas de red y brindé soporte técnico a los estudiantes, mejorando la eficiencia del equipo de TI.}

\cventry{2008 -- 2013}{Productor Musical}{Ableton}{Valencia, España}{}{Composición, producción, grabación, mezcla y masterización de pistas para artistas en Valencia, España.}

% Proyectos Destacados (junto con experiencia profesional)
\section{Proyectos}
\cventry{2023 -- presente}{Creador}{\href{https://github.com/ofou/graham-essays}{Graham Essays}}{Proyecto Open Source}{}{
\begin{itemize}
\item Desarrollé un programa que automatiza la creación de libros electrónicos (EPUB) a partir de contenido web.
\item Alcanzó la portada de Hacker News en su lanzamiento (\href{https://news.ycombinator.com/item?id=32465435}{ver discusión}).
\item Tecnologías: Python, web scraping, procesamiento de texto y generación de EPUB.
\end{itemize}
}

% Idiomas y Certificaciones (agrupados juntos)
\section{Idiomas}
\cvitemwithcomment{Español}{Nativo}{\cvskill{5}}
\cvitemwithcomment{Inglés}{C2 Proficiente}{\cvskill{5} EFSET 77/100}
\cvitemwithcomment{Chino Mandarín}{Básico}{\cvskill{1} En progreso}

\section{Certificaciones}
\cvlistitem{EFSET English Certificate C2 Proficient (77/100)}
% \cvlistitem{Deep Learning Specialization - Coursera/Andrew Ng}
% \cvlistitem{AWS Certified Machine Learning - Specialty}
% \cvlistitem{Google Cloud Professional Data Engineer}

\break

% Versión más sucinta del Skill Matrix
\section{Competencias Técnicas}

\cvskillhead[-0.1em] 

% Lenguajes y Tecnologías Core
\cvskillentry*[0.5em]{Lenguajes}{5}{Python}{6}{Desarrollo ML/AI, análisis de datos}
\cvskillentry{}{4}{SQL}{5}{Diseño y optimización de consultas}
\cvskillentry{}{4}{Bash/Shell}{5}{Automatización y scripting}

% Frameworks y ML/AI
\cvskillentry*[0.5em]{IA}{5}{RAG/LLMs}{3}{Implementación, fine-tuning, evaluación}
\cvskillentry{}{4}{PyTorch/HuggingFace}{3}{Modelos de deep learning}
\cvskillentry{}{4}{NLP}{4}{Análisis semántico, embeddings}
\cvskillentry{}{3}{LLM Evaluation}{2}{Métricas y benchmarks para modelos}

% Cloud y Datos
\cvskillentry*[0.5em]{Infra}{4}{Azure/AWS/GCP}{3}{Despliegue de soluciones ML/AI}
\cvskillentry{}{4}{MongoDB/PostgreSQL}{4}{Modelado y optimización}
\cvskillentry{}{4}{Git/Github}{5}{CI/CD, colaboración}
\cvskillentry{}{4}{Linux}{5}{Administración, customización}

% Herramientas y Frameworks
\cvskillentry*[0.5em]{Tools}{4}{LangChain/LlamaIndex}{2}{Aplicaciones con LLMs}
\cvskillentry{}{3}{Docker/Compose}{3}{Contenedores, orquestación}
\cvskillentry{}{4}{LaTeX/Jupyter}{4}{Documentación técnica}
\cvskillentry{}{3}{JS/React}{3}{Desarrollo frontend}

\section{Habilidades Blandas}

\cvdoubleitem{\textit{Resolución de problemas}}{\small Desarrollo soluciones innovadoras para desafíos de IA con enfoque analítico.}{\textit{Comunicación}}{\small Transmito conceptos técnicos complejos a audiencias diversas, creando contenido educativo viral.}

\cvdoubleitem{\textit{Colaboración}}{\small Trabajo eficazmente en equipos multidisciplinarios y distribuidos.}{\textit{Adaptabilidad}}{\small Me adapto rápidamente a nuevas tecnologías y entornos de trabajo remoto.}

\cvdoubleitem{\textit{Auto-\\aprendizaje}}{\small Adquiero continuamente conocimientos en tecnologías emergentes de IA y desarrollo.}{\textit{Creatividad}}{\small Genero ideas innovadoras y soluciones creativas para problemas técnicos complejos.}

\section{Referencias}

\cvitem{}{Disponibles a petición.}

\end{document}