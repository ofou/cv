\documentclass[11pt,letterpaper,roman]{moderncv}
\moderncvstyle{classic}
\moderncvcolor{blue}

\usepackage[scale=0.8]{geometry}

% Datos personales
\name{Omar}{Olivares Urrutia}
\title{Ingeniero de Inteligencia Artificial}
\address{Providencia, RM, Chile}
\phone[mobile]{+56~9~8381~0103}
\email{omar@olivares.cl}
\social[linkedin]{ofou}
\homepage{olivares.cl}
\photo[64pt][0.4pt]{photo.jpeg}

\begin{document}

\makecvtitle

% Resumen Profesional
\section{Resumen Profesional}
\cvitem{}{Ingeniero de Software en \href{https://www.emergentmind.com}{Emergent Mind}, especializado en Inteligencia Artificial, Machine Learning y aplicaciones. He implementado sistemas RAG y fine-tuning para mejorar la calidad de las respuestas de modelos de lenguaje. Como divulgador científico y creador de contenidos independiente (en inglés), he producido videos virales que han superado las 200.000 vistas sobre contenidos técnicos acerca de interfaces cerebro-computador.}

% Experiencia Laboral
\section{Experiencia Laboral}
\cventry{Febrero 2024 -- Presente}{Ingeniero de Inteligencia Artificial}{\href{https://www.emergentmind.com}{Emergent Mind}}{Carolina del Norte, Estados Unidos}{}{%
\begin{itemize}%
\item Diseño pipelines con RAG para mejorar la calidad y relevancia de los resultados de búsqueda en Azure.
\item Colaboro con el fundador en la estrategia y desarrollo del producto para la plataforma de descubrimiento de investigaciones en arXiv.
\item Implemento y optimizo modelos de aprendizaje profundo y modelos de lenguaje.
\item Realizo experimentos para optimizar el rendimiento y la precisión de los modelos.
\item Desarrollo evaluación y control de calidad para modelos de lenguaje que utilizan RAG.
\item Conceptualizo y prototipo características innovadoras de IA para impulsar el crecimiento de la plataforma.
\end{itemize}}

\cventry{Septiembre 2021 -- Enero 2025}{Desarrollador de Contenido Técnico}{\href{https://www.neurapod.com}{Neura Pod}}{San Diego, California, Estados Unidos}{}{%
\begin{itemize}%
\item Produje contenido sobre neurotecnología e IA, contribuyendo al crecimiento del canal a más de 80.000 suscriptores en YouTube.
\item Gestioné la producción completa: ideación, guiones, grabación de locuciones, edición y distribución de videos.
\item Traduje conceptos científicos complejos en contenido atractivo y accesible para audiencias amplias.
\item Creé múltiples videos virales con más de 200.000 vistas cada uno.
\item Expandí la presencia de Neura Pod en plataformas de redes sociales.
\end{itemize}}

\cventry{Julio 2021 -- Septiembre 2021}{Pasante}{\href{https://www.neurapod.com}{Neura Pod}}{San Diego, California, Estados Unidos}{}{Inicié mi carrera en creación de contenido técnico, aprendiendo las bases de la producción audiovisual y la comunicación científica.}

\cventry{2021}{Ingeniero de Software}{\href{https://www.rayo.com}{Rayo}}{Remoto}{}{Optimicé el uso de MongoDB y SQL, reduciendo costos operativos en un 15\% mediante técnicas avanzadas de indexación y caching.}

\cventry{2020}{Ingeniero de Software}{\href{https://www.suncast.cl}{Suncast}}{Las Condes, Región Metropolitana de Santiago, Chile}{}{Ayudé a reconstruir la plataforma en una arquitectura de microservicios, mejorando la escalabilidad y reduciendo el tiempo de respuesta en AWS.}

\cventry{2016 -- 2018}{Asistente de Soporte en TI}{\href{https://www.utalca.cl}{Universidad de Talca}}{Curicó, Región del Maule, Chile}{}{Resolví problemas de red y brindé soporte técnico a los estudiantes, mejorando la eficiencia del equipo de TI.}

\cventry{2008 -- 2013}{Productor Musical}{Ableton}{Valencia, Comunidad Valenciana, España}{}{Composición, producción, grabación, mezcla y masterización de pistas para artistas en Valencia, España.}

% Educación
\section{Educación}
\cventry{2018 -- 2024}{Ingeniero Civil Informático}{\href{https://portal.ucm.cl}{Universidad Católica del Maule}}{}{}{\textit{Especialización en procesamiento de lenguaje natural e Inteligencia Artificial.} Desarrollé un fine-tuning del modelo de lenguaje gpt-3.5-turbo para ayudar a los estudiantes de la PAES a través de tutorías personalizadas como tesis de pregrado. Nota 7.0}
\cventry{2015 -- 2018}{Ingeniero Civil en Computación}{\href{https://www.utalca.cl}{Universidad de Talca}}{}{}{\textit{Enfoque en arquitectura de computadores, algoritmos y desarrollo de software.} Estuve estudiando algunos años en la Universidad de Talca, pero por motivos familiares me cambié a la Universidad Católica del Maule.}

% Competencias Técnicas
\section{Competencias Técnicas}
\cvitem{\textbf{Lenguajes}}{Python, SQL, JavaScript, LaTeX}
\cvitem{\textbf{Frameworks}}{PyTorch, Hugging Face, React, Tinygrad, MLX}
\cvitem{\textbf{Tools}}{Git, Docker, CI/CD}
\cvitem{\textbf{Big Data}}{Azure, GCP, AWS, MongoDB, PostgreSQL}
\cvitem{\textbf{IA/ML}}{RAG, LLMs, NLP, Computer Vision, Langchain, Streamlit}
\cvitem{\textbf{Otros}}{Jupyter, Linux}

% Proyectos Destacados
\section{Proyectos Destacados}
\cventry{2023 -- presente}{Creador}{\href{https://github.com/ofou/graham-essays}{Graham Essays}}{Proyecto Open Source}{}{Desarrollé un programa que ayuda a la creación de un libro electrónico (EPUB) a partir de una página web de manera automatizada. Llegó al frontpage de Hacker News en su lanzamiento https://news.ycombinator.com/item?id=32465435}

% Idiomas
\section{Idiomas}
\cvitemwithcomment{Español}{Nativo}{}
\cvitemwithcomment{Inglés}{C2 Proficiente}{EFSET 77/100}
\cvitemwithcomment{Chino Mandarín}{Básico}{En progreso}

% Certificaciones
\section{Certificaciones}
\cvlistitem{EFSET English Certificate C2 Proficient (77/100)}
% \cvlistitem{Deep Learning Specialization - Coursera/Andrew Ng}
% \cvlistitem{AWS Certified Machine Learning - Specialty}
% \cvlistitem{Google Cloud Professional Data Engineer}

% Intereses
\section{Intereses}
\cvitem{}{Producción musical, contribuciones a proyectos de código abierto, aprendizaje de mandarín, viajes, neurotecnología e interfaces cerebro-máquina.}



\end{document}